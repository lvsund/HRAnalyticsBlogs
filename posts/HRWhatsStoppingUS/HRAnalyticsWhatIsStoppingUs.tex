% Options for packages loaded elsewhere
\PassOptionsToPackage{unicode}{hyperref}
\PassOptionsToPackage{hyphens}{url}
%
\documentclass[
]{article}
\usepackage{amsmath,amssymb}
\usepackage{iftex}
\ifPDFTeX
  \usepackage[T1]{fontenc}
  \usepackage[utf8]{inputenc}
  \usepackage{textcomp} % provide euro and other symbols
\else % if luatex or xetex
  \usepackage{unicode-math} % this also loads fontspec
  \defaultfontfeatures{Scale=MatchLowercase}
  \defaultfontfeatures[\rmfamily]{Ligatures=TeX,Scale=1}
\fi
\usepackage{lmodern}
\ifPDFTeX\else
  % xetex/luatex font selection
\fi
% Use upquote if available, for straight quotes in verbatim environments
\IfFileExists{upquote.sty}{\usepackage{upquote}}{}
\IfFileExists{microtype.sty}{% use microtype if available
  \usepackage[]{microtype}
  \UseMicrotypeSet[protrusion]{basicmath} % disable protrusion for tt fonts
}{}
\makeatletter
\@ifundefined{KOMAClassName}{% if non-KOMA class
  \IfFileExists{parskip.sty}{%
    \usepackage{parskip}
  }{% else
    \setlength{\parindent}{0pt}
    \setlength{\parskip}{6pt plus 2pt minus 1pt}}
}{% if KOMA class
  \KOMAoptions{parskip=half}}
\makeatother
\usepackage{xcolor}
\ifLuaTeX
  \usepackage{luacolor}
  \usepackage[soul]{lua-ul}
\else
  \usepackage{soul}
\fi
\setlength{\emergencystretch}{3em} % prevent overfull lines
\providecommand{\tightlist}{%
  \setlength{\itemsep}{0pt}\setlength{\parskip}{0pt}}
\setcounter{secnumdepth}{-\maxdimen} % remove section numbering
\ifLuaTeX
  \usepackage{selnolig}  % disable illegal ligatures
\fi
\IfFileExists{bookmark.sty}{\usepackage{bookmark}}{\usepackage{hyperref}}
\IfFileExists{xurl.sty}{\usepackage{xurl}}{} % add URL line breaks if available
\urlstyle{same}
\hypersetup{
  hidelinks,
  pdfcreator={LaTeX via pandoc}}
\usepackage{hyperref}
\title{HR Analytics -- What Stopping Us And Where Do Go From
Here?}
\author{Lyndon Sundmark}
\date{Dec 29, 2023}

\begin{document}
\maketitle


\section{Introduction}\label{introduction}

~

Earlier this year I wrote a few blog articles on HR analytics. In one of
those articles I indirectly expressed doubts, concerns, and readiness of
the HR profession to capitalize on the promise of HR analytics and take
rightful claim and ownership of it within the services that HR provides.

\href{https://www.linkedin.com/pulse/why-hr-might-able-reinvent-itself-lyndon-sundmark-mba?trk=mp-reader-card}{The Article}



I mentioned a number of obstacles that might exist-historical and
current- that could be standing in the way. Many of these are related to
how HR views itself as a profession and a lack of readiness to see
itself as a far more technical field and profession.

My sense is, though, that even if there was an immediate willingness by
HR to embrace analytics and its `technical' HR side, there would still
be some immediate obstacles that would prevent uptake by HR.

\section{What is the hold up?}\label{what-is-the-hold-up}

There is still a problem of `where we start'. Most people, to adopt new
methodologies, require a `show me'. That `show me' has to be in the
context that is relevant to them. In this case with HR, the context has
to be HR examples. In any new developing field, there is often slow
uptake on adoption of methodology because of a dearth of `show me'
examples.

~There are at least three essential ingredients to a `show me' for HR
analytics;

\begin{itemize}
\item
  Collect data that illustrates what we are attempting to convey
\item
  Show how to apply data science/ predictive modelling and HR analytics
  methodologies to it
\item
  Share that with the HR community.
\end{itemize}

So what's the problem? We have oodles of HR data. Analytical
methodologies are bursting on to the scene. The ingredients are there to
do it, but there is little if any action.

Part of the problem is that there is interdependency of data needed and
analytic procedures. What procedures we use will dictate the data we
need and the form it needs to be in. We may have `data' but it may not
be the data we need. And how we know which analytical technique to use?
We can't collect potentially relevant useful data for analysis, until we
decide on the analysis. And we can't decide on the proper analysis until
we understand what are analytical options are. And understanding those
usually requires data to work with.

This presents us with a bit of a dilemma- analytical procedures require
data, in order for us to learn. But the relevant data being able to be
gathered being dependent on the procedure being chosen. And none of this
being able to be shared with HR community until we have workable HR
examples. And to encourage the forward movement of analytics in HR - we
need those examples.

~

~

~

~

~

\section{Resolving the Dilemma}\label{resolving-the-dilemma}

In effect, the circular dependency of data and analytical examples has
to be broken and a start has to occur somewhere. I think that a start
can occur with an identification of a business problem that has HR
implications, AND an inquiry.~\textbf{`Is this an HR business problem
that HR analytics can address and help solve'?}~If we are courageous
enough to ask the question, and if it truly requires an answer by the
business we are likely to start the process of data gathering/collection
and research to answer the question.

In fact if we truly accept some of the common definitions for workforce,
HR, People Analytics

See~\href{https://www.linkedin.com/pulse/workforcehrpeople-analytics-hr-lyndon?trk=mp-reader-card}{The Article}

we must ask the above question. Dr.~John Sullivan's definition of People
Analytics is particularly illustrative of this

~

\emph{From Dr.~Sullivan's article:}

\href{http://www.eremedia.com/tlnt/how-google-is-using-people-analytics-to-completely-reinvent-hr/}{Dr Sullivan's Article}

\emph{``\textbf{People analytics}~is a data-driven approach to managing
people at work. Those working in people analytics~\textbf{strive to
bring data and sophisticated analysis to bear on people-related issues},
such as recruiting, performance evaluation, leadership, hiring and
promotion, job and team design, and compensation \ldots{}}

~

Data driven approaches, sophisticated analyses etc. require us to think
outside the box and have the~\textbf{spirit of inquiry regarding data
and analytical methods}~that can be brought to bear.

But even if we have the data, how do we begin to get our head around
what are option for sophisticated analyses?

\section{The Basics}\label{the-basics}

~I think there are some basic building blocks of understanding that can
be tremendously useful. These building blocks cover:

\begin{itemize}
\item
  The Role and Understanding of Measurement
\item
  The Importance of Measurement Scope
\item
  The Purpose and Role of Statistics and Statistical Analysis to HR
\item
  The Role of Statistical Software, Data Science /Machin Learning/Predictive Modelling to HR Analytics
\end{itemize}

~

\subsection{Measurement}\label{measurement}

It begins with the understanding the definition and the role of
measurement. It's important to understand that measurement is woven into
the very fiber of HR DNA whether we acknowledge this consciously or not.
Let's look at a definition of `measurement'

From~\href{https://en.wikipedia.org/wiki/Measurement}{\ul{https://en.wikipedia.org/wiki/Measurement}}

\textbf{Measurement}~is the assignment of a number to a characteristic
of an object or event, which can be compared with other objects or
events.

You could probably substitute `description' for `characteristic' in the
above definition too.

When you think of above definition, the very act or recording HR
information into information systems is `measurement'. This is because
to store the information into the system we assign numbers or
descriptions/categories to it. We have employee `numbers', gender, age,
birthdates, names, titles, salaries etc. These are either numeric pieces
of information or descriptions of things or categories.

That definition says nothing about the rules of measurement, or how good
the measurement is. These are important concerns too, but at the most
basic level, once we are storing information we are
measuring.~\textbf{So much for the argument that HR is a non-technical
field}. And why do we store information and `measure'? - To describe and
understand end explain the world around us.

If we embrace the reality that HR too is about measuring-we have one of
the first building blocks of heading in the direction of sophisticated
analyses mentioned above.

\subsection{HR Measurement Scope}\label{hr-measurement-scope}

Once we understand the centrality of measurement to what we do in human
resources management, it's important to get our heads around the scope
of what can be measured. Indeed, what can be measured is almost
limitless. What can we, should we be paying attention to? And once we
decide on what to measure, how or where does it stand compared to other
things we measure? How do we see and keep track and keep sense of that
we measure?

One possible framework, and there could be others, that could be useful
is one I introduced earlier this year in the following article

\href{https://www.linkedin.com/pulse/what-does-data-driven-hr-look-like-lyndon-sundmark-mba?trk=mp-reader-card}{What Does Data-Driven HR Look Like?}

~

In it, I had suggested that `metrics', the things we measure, typically
fall into 1 of 3 categories for HR purposes:

\begin{itemize}
\item
  HR Activity

  \begin{itemize}
  \tightlist
  \item
    These metrics describe what is going on with the people in our
    organization. Examples are counts of current employees, turnover
    counts, hire counts, turnover and hiring rates, absenteeism,
    accident and injury rates, benefits participation rates, training
    enrollments, employee churn, grievances counts and rates. If it
    describes what is going on with our employees and what they are
    doing, or what is happening to them they are in this category
  \end{itemize}
\item
  HR Process Efficiency and Effectiveness

  \begin{itemize}
  \tightlist
  \item
    These metrics concern themselves with HR business processes, the
    demand for them, the length of time to provide HR services within
    our HR business processes, customer feedback on HR Services, waiting
    time to receive service. They aren't about people activity which
    makes it different than the category above. They are about measuring
    HR services or business activity.
  \end{itemize}
\item
  HR Methodologies

  \begin{itemize}
  \tightlist
  \item
    These are metrics or measures which are generated or used or could
    be used in the actual methodologies we use to carry out our HR
    functions or services. It's when we define the goal of our function
    or service and try to improve on reaching the goal, have a better or
    more accurate outcome in the goal and changing our methodologies or
    processes to improve the outcome. And its measuring what we are
    striving for and along the way to getting there. The purpose is to
    do what we are doing even better. It might include measures
    necessary for more accurate job classification, better sourcing of
    candidates in recruitment, better screening of candidates, and
    better selection of leaders within the organization.
  \end{itemize}
\end{itemize}

If we have a framework to see our HR measurement in, its becomes a whole
lot easier to see what we haven't done, what we have done, and what we
could do with respect to bringing sophisticated analyses to bear in HR.
Having a framework is a second good building block

~

~

~

\subsection{\texorpdfstring{\textbf{Purpose of
statistics}}{Purpose of statistics}}\label{purpose-of-statistics}

Even though statistics, statistical methodology and statistical software
and tools have been around for decades and predate PCs, a lot of people
misunderstand statistics and statistical analyses and their purpose.

Perhaps this is because in most people's minds and experiences, when you
mention statistics or statistical analyses, the first thing than comes
to mind is a boring summary table of numbers or confusing business
chart, or summary data on a spreadsheet. I think data warehousing tools,
unintentionally, also continue to foster this misunderstanding. We are
simply given more and easier tools to slice and dice sometimes what
appears to be boring business data. The predominance of introduction to
charts and summary data and use of spreadsheet for this purpose in
financial applications doesn't help this either.

If, in the previous section, at the heart of all data collection is
measurement (without mentioning what we are measuring for) - then at the
heart of statistical analyses and methodology is understanding the world
around us. We want to answer the question of what are we measuring for:
inevitably to provide an answer or solution to a business question or
problem.

\textbf{Yes, at its most basic level, statistics and statistical
analyses is about descriptive statistics, such as counts percentages,
and averages.~ But it is so much more than that}. While it would take a
statistics course to fully understand the typical repertoire of
statistical procedures at our disposal and what they do, their purposes
fall into probably just a few broad categories:

\begin{itemize}
\item
  \textbf{Describing what we have or are looking at.}~This is the
  purpose of descriptive statistics and charts. Much HR data analysis in
  organizations doesn't go beyond this- either because the organization
  doesn't demand it, or the skills are present in HR to do more.
  Examples here are employee counts, hiring counts, termination counts
  etc.
\item
  \textbf{Examining relationships in our data}. How different things we
  are measuring are related to each other. If we have information on
  employee age and absenteeism rates , do our measurements show these
  two things to be related? If so, does absenteeism go down with age, or
  up? Or is there no discernable relationship?
\item
  \textbf{Determining whether what we are seeing in our data is
  universal right across the organization}. This could apply to both
  descriptive statistics and relationship. Is something universally true
  or does it vary across/between the categorical things we
  capture/measure? In the above example, does absenteeism vary by
  gender, or organizational level, or by job.
\item
  \textbf{Determining whether differences we see are statistically
  significant}. Are the differences we are seeing attributable to the
  different categories or could they be due to random chance?
\item
  \textbf{Predicting accurately something that we don't or have from
  something we do know or have-concurrently or in the future.} This is
  an extremely important consideration in business. To the extent to
  which we can predict the future based on what our past data and
  history tell us, allow us to pro-act and to a certain extent actively
  manage our future.
\item
  \textbf{Understanding whether what we are measuring is related to
  time, and if so how?}~There is a whole category of statistical
  methodologies, known as time series analysis, who purpose if to answer
  that question- teasing trend, seasonal, cyclical fluctuations out of
  our data. Stock market data is a good candidate example of this type
  of data and analysis.
\item
  \textbf{Predicting the best fit category for something for a new item
  where categories preexist for existing items.}~In HR, a perfect
  example of this is job classification. A new item (job) needs to be
  classified into a job level and/or job family. Other jobs are already
  classified to family or level. But we now have a new job that has been
  created, and it needs to be classified fairly as to level and family
  based on its similarities and differences to other jobs in the same of
  different classifications. By the way, statistics and statistical
  analyses are a natural fit for this HR function, but have been in very
  little organizational evidence over the decades.
\item
  \textbf{Determining the optimal number of categories for some data
  where categories don't preexist}. Are their natural patterns in our HR
  data that help suggest where groups or categories naturally occur? In
  HR, sometimes salary broad banding is an example of this. We may have
  had previous categories, but they were no longer clearly
  differentiating well. So what does our data itself tell us about
  naturally forming groups or categories?
\item
  \textbf{Determining the upcoming choices people will make, based on
  past choices}. In HR an example might be open enrollment for benefits.
  Looking at past choices people have made, a virtual grocery cart or
  checkout, predict what future choices they might make on complementing
  benefits etc. These are normally called `recommender' systems
\end{itemize}

This isn't intended to exhaustive, but it does summarize a wide swath of
purposes that statistical analyses are intended to serve.~\textbf{If we
are ever going to reach the promise of HR analytics and sophisticated
analyses we have to understand that statistics, and statistical analyses
are our servant and tools in this picture. And we have to envision how
we can use them. This isn't optional.}

~

\subsection{\texorpdfstring{\textbf{Role of Data Science /Machine
Learning/Predictive Modelling/Data
Mining}}{Role of Data Science /Machine Learning/Predictive Modelling/Data Mining}}\label{role-of-data-science-machine-learningpredictive-modellingdata-mining}

From the previous two sections, if you accept the importance that
measurement and statistical analysis play in HR analytics and
sophisticated analysis, then you will understand more clearly the role
of data science. Machine learning, predictive modelling and data mining
play. These methodologies are very much statistical analysis
methodologies. But they are more than that because they attempt to more
formalize a methodology around an `inception to completion' analysis
approach rather than just a specific type of statistical analyses or
procedure.

These emerging methodologies all have a common denominator
--~\textbf{finding patterns in data}. The desire for looking for
patterns in data come from wanting to solve or understand a problem. In
the context of HR, it comes from wanting to solve a business problem or
create new opportunities.

There are no guarantees up front, that there will be patterns in our
data that are `business meaningful' and relevant. But there no guarantee
either that the data we collect is meaningless. We have to make the
decision to look at it with these powerful methodologies and tools.

It would probably take several books to go into detail on these
methodologies, and that isn't the purpose of this blog article. But I
will try to illustrate further their relevance, first of all with web
links for books, sites, training etc., and then with some examples of
how these could be used in HR. While I will share some examples, my
intent is to really spur HR into further creative thinking on
application of these methodologies to HR business problems.

\textbf{Some links}

-There is no particular importance attached to these as compare to many
others. They just happen to be ones I am familiar with.

\textbf{Data Science and Machine Learning}

I have been at the time of writing this blog article been taking a free
online edX course

\href{https://www.edx.org/course/data-science-machine-learning-essentials-microsoft-dat203x}{data-science-machine-learning-essentials}

This covers Microsoft's state of the art tool for machine learning
--AzureML. They have dozens of example templates that give an idea of
the possible uses on machine learning to find patterns. I will share
some of these shortly -- below.

~

\textbf{Predictive Modelling}

A book that I have come across in my travels for predictive modelling is
Applied Predictive Modelling by Kuhn and Johnson

\href{http://www.amazon.ca/Applied-Predictive-Modeling-Max-Kuhn/dp/1461468485/ref=sr_1_1?s=books&ie=UTF8&qid=1445460069&sr=1-1&keywords=applied+predictive+modeling}{Applied-Predictive-Modeling-Max-Kuhn}

~

If uses the R statistical language and software and its `caret' package

~

\textbf{Data Mining}

\href{http://www.amazon.ca/Predictive-Analytics-Data-Mining-RapidMiner/dp/0128014601/ref=sr_1_2?s=books&ie=UTF8&qid=1445460564&sr=1-2&keywords=rapidminer}{Predictive-Analytics-Data-Mining-RapidMiner}

~

This book covers Rapidminer- which is software specifically designed for
data mining.

I have used all 3 of these at one point or another. Again there are many
other software packages and books out there. These are 3 I have come
across.

~

\textbf{Patterns in the Data}

So if all of these methodologies have at their `heart' finding patterns
in the data, why should we care about this? The primary reason is
prediction- whether concurrently or for the future. Sometimes being able
to predict allows us to solve a business problem (including HR related)
or help us to discover new opportunities.

To give an idea of the possibilities, a perusal of the sample experiment
templates in AzureML gives us a glimpse of the possible application of
machine learning

\href{http://gallery.cortanaanalytics.com/browse/?categories=\%5b\%22Experiment\%22\%5d&examples=true}{Cortana}

You may have click for `new' experiment to see existing examples.

Looking at the keyword for samples, finding patterns in the data can
take the form of:

\begin{itemize}
\item
  Classification of items into categories
\item
  Anomaly Detection
\item
  Clustering like with like
\item
  Fraud detection
\item
  Predicting/forecasting -- using regression
\item
  Predicting/forecasting --using time series analysis
\item
  Predicting/forecasting - churn/turnover
\item
  Text Classification
\end{itemize}

We need to envision how these might apply to various functions in HR

\begin{itemize}
\item
  Classification is a `natural' for job classification- in terms on
  intent
\item
  Anomaly detection- might be useful in measuring the
  performance/delivery of our HR services to our clients. Did any
  service take abnormally long to deliver that shouldn't have? How would
  we know this? Anomaly detection
\item
  Clustering- might be useful for salary broad banding- creating new job
  families and levels based on characteristics of similar work.
  Clustering like with like
\item
  Predicting/forecasting using regression -- determining competitive
  salaries for positions that are not similar to key comparators.
\item
  Predicting/forecasting --using time series analysis- being able to
  plan ahead of time for the cost of maternity leaves
\item
  Predicting/Forecasting churn or turnover. In retail sales, business
  are concerned with customer churn- those that will leave them for
  competitors. Incentivizing the customer sometimes will prevent that.
  For HR, does predicting who will turnover (especially roles and people
  you don't want to lose) merit the organization attention? Is it
  important or not?
\item
  Text Classification- suppose you had historical data on successful
  hires and not successful hires. Suppose you had their original
  resumes. Supposed there was a way to discover patterns in the text of
  the resumes? Suppose that those patterns were predictive of those who
  were successful hire and those that weren't? Is what worth something
  to organizations?
\item
  Recommending `like' or `similar' services to employees based on
  current or past choices. These would be in category of recommender
  systems.
\end{itemize}

As was indicated previously, there are no guarantees of patterns in the
data and its predictive value. It's either there in the data or it
isn't. But if we aren't looking, how will we know? If we aren't looking
and our competitors are (yes HR has competition) are we at a competitive
disadvantage in the HR marketplace?

Unabashedly, my motive here is to stir HR into thinking about these
technologies and their application in their `world', pick up familiarity
with the tools, and start examining their data. As HR experts
what~\textbf{other areas}~in your HR business have application for this?
Is training and development an area? What about health and
safety/workers compensation?

~

\section{The Importance of Sharing
Examples}\label{the-importance-of-sharing-examples}

As indicated above, the technologies are there to do sophisticated
analysis in HR/People/Workforce Analytics. ~And as indicated previously,
lots of HR data exists. But it is the right kind of data? Is it tailored
to the business questions we are asking? Are we sharing examples so that
other can learn?

I think the correct terminology here might be `open' data.~ I recently
came across the following article on it, describing it in another
context.

~

\href{http://www.zdnet.com/article/open-data-open-mind-why-you-should-share-your-company-data-with-the-world/}{\ul{http://www.zdnet.com/article/open-data-open-mind-why-you-should-share-your-company-data-with-the-world/}}

~

~Traditionally, and rightly so, much HR information is confidential. And
it's also rightfully protected by privacy legislation. I guess the
question is, is it feasible, legal, and possible to share out HR data on
an anonymous basis that doesn't violate a person's privacy and
contravene privacy legislation? Is it possible to create HR datasets for
which:

\begin{itemize}
\item
  Others from the HR community can learn from and practice on to do
  proof of concepts for their own organizations?
\item
  The additional sets of eyes beyond the organization, may be
  beneficial?
\end{itemize}

~

This already exists in other fields of endeavor and data. An example of
this is Kaggle

\href{https://en.wikipedia.org/wiki/Kaggle}{\ul{https://en.wikipedia.org/wiki/Kaggle}}

~

Could this apply to HR as well? Can we move HR Analytics ahead by
sharing data and real life data science/ predictive modelling, data
mining examples? Should a website be created for people to upload
examples of their HR data science projects to share knowledge with
others? What say you?

\end{document}
