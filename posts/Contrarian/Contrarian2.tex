\documentclass[12pt,letterpaper]{article}
\usepackage[T1]{fontenc}
\usepackage{parskip}
\usepackage{hyphenat}
\usepackage{url}
%\usepackage{titling}
\title{People/HR Analytics - What Is your Paradigm ? It Matters}
\author{Lyndon Sundmark}
\begin{document}
\maketitle

\emph{TL;DR; People/HR Analytics long term sustainability into the
future does not rest on continuing to promote and treat it as an add-on
to HR but on the fundamental retooling of HR practices, methodologies,
and education within the HR profession itself so that it becomes the DNA
of how we `do' HR.}

\section{Introduction}\label{introduction}

Its been about 10 years since I started to see things written about HR
Analytics / People Analytics / Workforce Analytics - whichever term you
choose to use.

Much has happened in these last 10 years which has been incredibly
positive:

\begin{itemize}
\item
  There has been much written in this area -both blog articles and
  books.
\item
  Global and local conferences in this area have been going strong for
  the last 10 years and continue to do so.
\item
  There have been innumerable job postings by the major job boards for
  HR/People/Workforce Analytics roles at all levels including leadership
  levels in organizations.
\end{itemize}

All of this has served to hugely increase the visibility of HR/ People /
Workforce Analytics. Its gone from an unknown to wide visibility.

Enough time has passed and enough has been written to give a sense of at
least two paradigms that have emerged for People/HR Analytics. By
paradigms I mean:

\begin{itemize}
\item
  ways of defining it
\item
  the assumptions you make of it
\item
  how you might operationalize it
\item
  what you expect from it.
\item
  how you define and measure success in it.
\end{itemize}

The choice of your paradigm, in many ways, will determine the success
and long term sustainability of your efforts. So your choice is
important.

\section{Where are we in 2023?}\label{where-are-we-in-2023}

My sense is that at least these two paradigms have emerged:

\begin{itemize}
\item
  People/HR Analytics is seen an add-on to the discipline or field of HR
  requiring its own team, resources,expertise and function with the rest
  of HR remaining its traditional self.
\item
  People/HR Analytics is seen as a fundamental rethink and retooling of
  HR methodologies/practices/services of `traditional HR' to be
  `data-driven' and `evidence-based' (top to bottom).
\end{itemize}

Both of these share the recognition of the importance of HR information
and its analysis. But beyond that they differ in many ways.

The first of these seems to be the current `goto' to show evidence and
credibility of People /HR Analytics being in existence in organizations.
However, this paradigm may unnecessarily and unintentionally limit both
the scope of People/HR Analytics activities and its ultimate long term
success and sustainability. The second paradigm will likely dramatically
alter the HR landscape and increase the likelihood of successes and
longevity. It will be more challenging to achieve.

Unfortunately, too much of what is being currently written about
People/HR Analytics currently seems to be reflecting the first paradigm.
And not enough is being written regarding the second paradigm. Its
almost like there is an `echo chamber' present where the same thinking
gets repeated over and over and then becomes firmly entrenched in `best
practices' thinking.

In the rest of this blog article, I would like to explore further these
two paradigms and their possible implications. As I hinted at earlier, a
paradigm often reflects your definitions and terminology, the
assumptions you make, how you operationalize your efforts, what outcomes
you expect and how you measure your success.

\subsection{What Typifies The `Add-On' Paradigm
?}\label{what-typifies-the-add-on-paradigm}

In the `add-on' paradigm, the way in which people / HR analytics is
defined usually as an activity separate from either existing HR
functions or activities. It sees it as `gathering employee information
and carrying out analysis and processing of that information with the
hopes that this information will impact business outcomes'.

The way in which it is suggested to be operationalized in organizations
is to create a specially functioning HR analytics team charged with the
responsibility of making breakthrough discoveries that will impact
business outcomes. This team is distinct from all other `traditional'
functional HR teams.

To operationalize this often requires a business case to be made for the
resources to bring this team into existence- since it is perceived to be
an `add-on' function to the traditional functions that already exist in
HR. In most cases-inherent in that business case (and the hopefully
subsequently obtained resources) is an implied expectation that this
investment of additional resources will pay off. In effect the analytics
team needs to `produce' and `keep on producing' to retain its existence.

Some of the first deliverables are the automation of HR metrics through
the use of Business Intelligence technologies and tools- including data
warehousing and dashboards. Often these initiatives make the provision
and visibility of HR information to decision makers in organizations
both efficient and timely for the first time. For some organizations
they equate these initiatives as being the whole of People / HR
Analytics. These initiatives can be terribly alluring because it is both
a tangible and visible deliverable. Its the starting point of HR
Analytics in many organizations and often it should be.

Once these are in place, the attention then often focuses then on what
kinds of questions are we able to look at and answer with this
information that would have been too cumbersome or impossible before.

The expectations are that this team will get the attention of executive
decision makers with breakthrough insights that will fundamentally
improve and optimize business performance, and that the team will keep
on delivering these insights.

In fact the criteria for success in this paradigm will often be exactly
that- continual delivery of new breakthrough insights. After all, scarce
organizational resources have been committed and this team must keep on
delivering and impressing or its `why did we do this in the first
place?' time. The team must earn its keep and keep on jusstifying its
existence.

You can tell by the description above, that everything about this
paradigm shouts 'add-on'. From its definition, to its assumptions,
operationalization , expectations, and measures of success.

\subsubsection{How does this contrast with the alternate paradigm?}

In the alternate paradigm, People/ HR Analytics is not seen as a
separate activity or function of HR.

In this paradigm, People/ HR Analytics is defined as `data-driven',
`evidence-based' HR Management and Decision Making. This means several
things:

\begin{itemize}
\item
  Its scope covers ANYTHING related to HR Management and Decision
  Making.

  \begin{itemize}
  \item
    HR management and decision making is impacted by the 
    analysis ofdata generated by the interaction of people 
    with the organization - their onboarding (applicants), 
    their interactions while within the organization,
    and their exit (terminations). BUT it also includes
    the scope of \textbf{ALL} HRIS information. 
    This is much more than just employee and applicant information.
  \item
    HR management and decision making is ALSO impacted by \textbf{how
    well we conduct the operations of business of HR - the provision of
    HR Services.} This means looking at ourselves as a business within a
    business -with our own internal customers and collectiing and
    analyzing information on the provision of those services.
  \item
    Finally, HR management and decision making is impacted by the degree
    to which we pay attention to improving the very practices and
    methodologies we use in HR injecting `data-driven, evidence based'
    into these wherever possible to improve the accuracy, reliability of
    them and to enhance the explainability of and trust in their
    results.This too requires the generation and analysis of data. We
    concern ourselves not only with what we do and how well we do it,
    but on exactly how we do it and improve that. We seek to leverage AI
    and machine learning to assist and change how we do what we do.
  \end{itemize}
\item
  This paradigm does not assume a separate `add-on' function. In fact it
  recognizes /sees People /HR Analytics as a change in and a choice of
  how we conduct the business of HR we are already in- HR Management and
  Decision Making which are `data-driven', `evidence-based'.
  Historically, HR decision making and management have been notoriously
  subjective and non data driven, with possibly the exception of
  information used in collective bargaining, salary surveys, and
  workforce forecasting/ planning where practiced. In most recruitment
  activities, performance appraisal, job descriptions -data drivenness
  evidence-based has been minimal or non-existant.
\item
  This paradigm does not automatically assume that extra resources will
  be necessary to engage in data-driven, evidence-based HR Management
  and Decision Making. Initially much of this paradigm can be
  operationalized within existing resources.
\end{itemize}

\section{Why is this distinction so
important?}\label{why-is-this-distinction-so-important}

When we see People Analytics / HR Analytics as the second paradigm (data
driven and evidence based HR management and decision making - the HR
management and decision making part of this is something that already
exists in organizations.

Making it `data driven' and `evidence based' in terms of our HR
practices -is simply a choice. \textbf{A choice that HR makes as to how
it conducts the business of HR.} HR can choose to do this or `not' -
both in an organization in terms of its practices, or as an HR
profession as to how we tool and retool ourselves for the future.

As HR - we don't ask for `outside permission' to do this. We either
recognize the contributions that data driven and evidence based make to
the quality and performance of our HR services and do this- or we don't.

The start of the HR Analytics journey doesn't require a big resource ask
or the selling of a product or service in search for a need or need for
recognition.

For example - If you have an HR issue that data might solve, and you
either have the data or can get it, and have the knowledge/skills in
statistics and machine learning- you have the ability to be data driven
and evidence based in your decision making on a standalone basis. The
statistical tools can be free, and these days your access to computers
and processing power on your desktop is not a barrier either. Period.
This doesnt require an extra team and resources to start showing
evidence of data-driven and evidence-based.

How much this effort or activity can be organizationally `enhanced'
however might be a function of additional resources.

\section{What Happens When We Adopt The First
Paradigm?}\label{what-happens-when-we-adopt-the-first-paradigm}

Several of the following may occur:

\begin{itemize}
\item
  We tend to define People/HR Analytics based on our immediate
  organizational needs rather than on a standardized set of taxonomies,
  concepts and definitions
\item
  We focus on the wrong decision makers and on the wrong things for the
  wrong reasons.
\item
  We define our successes or failures the wrong way
\item
  We start thinking in terms of territoriality
\end{itemize}

Let look at each of these in detail.

\textbf{We tend to define People/HR Analytics based on our immediate
organizational needs rather than on a standardized set of taxonomies,
concepts and definitions}

On a personal level, when I seek to understand a field or body of
knowlege, I want to understand it in a manner where I can see it both
from a whole/totality perspective and its parts when and where
necessary. For me then - how we define it is fundamentally important.
Its definition has to be broad enough to achieve the above.

As HR we often allow consultants and vendors to define it in terms of a
product or service that they sell. Inevitably- doing that causes it to
be defined in terms of part of the picture rather than the `whole'.

\begin{itemize}
\item
  Some have defined it as simply the collection and analysis of employee
  data.
\item
  Some have equated it to HR metrics. If you have these then they claim
  you are doing HR analytics.
\item
  Some claim that pre-packaged software they have created is HR
  Analytics software, and if you purchase it you are doing HR Analytics.
  (ie no need to take responsbility for it -we do it for you)
\item
  Some mistakenly equate HR data warehousing/tools/initiatives as HR
  Analytics.
\end{itemize}

All of these narrowly define it to operationalize it to something that
they are selling.

This may be good for their business or product, but works against a
clear understanding and adoption of HR Analytics by HR professionals and
HR functions within organizations.

It leads to confusion in terms of when we are dealing with just a part
of the whole versus the whole. We may have expectations of success for
the whole, when in fact we are doing just a part. The `part' will not
meet the expectations of the `whole'

Definitions, concepts and taxonomies in People Analytics should be
driven by the HR profession in conjunction with unversities and
educational institutions. Vendors and consultants should understand
these and should be able to explain how `their' products or services fit
into that , and what `part' of the overall picture they address within
it.

\textbf{We focus on the wrong decision makers and on the wrong things
for the wrong reasons.}

There have been articles written about People/HR Analytics on how to get
the C suite to pay attention to this and to get the resources to do it.
Or even worse complaining `why HR isn't at the executive table'- and
feeling that People Analytics will finally be their key to this table.

This happens in part, because we see People /HR Analytics as something
`in and off itself'- separate from the rest of HR functions. Something
that requires its own resources, life , and existence. Something that
needs to prove itself to remain in existence or to be brought into
existence in the first place. Something that cant be done without
additional resources.

We chase after extra resources (along with every other area of the
organization). We have to focus on the C suite to get them. To request
additional resources for something completely new- we end up needing to
make a business case. This is often very difficult if not impossible in
these circumstances. Ofteni in the time it takes to make a business case
to a skeptical executive, you could have developed a prototype of an
application demonstrating its use. Its often easier to get additional
resources for something that pre-exists prior to the request.
Personally, over my own career, I saw far too many significant business
breakthough ideas die on the altar of the `business case' gods.

When we finally see People/HR Analytics as changing the way we do HR-
then our focus shifts:

\begin{itemize}
\item
  off of the c-suite
\item
  off of the business cases
\item
  off of the need at least initially for extra resources.
\end{itemize}

We finally recognize that the right decision makers we have to focus on
is ourselves -HR from top to bottom. Its up to us to make it happen -
not anyone else- and we dont need outside permission to do this.

The only people that can re-tool HR into `data-driven' and
`evidence-based' management and decision making is HR itself. No amount
of new resources will make any difference without HR itself
fundamentally thinking differently.

Which brings me to the other point above about HR complaining about not
being represented in the C - suite. I think that's the wrong focus. The
premise itself can often come with baggage of inferiority, because we're
not good enough to be represented there. That somehow HR needs to
justify its existence and worthiness to be there. Consider the
following:

\begin{itemize}
\item
  Can organizations manage effectively without their financial
  representation, resources and professionals at the executive table?
\item
  Can they without their operational representation, resources and
  professionals?
\item
  Can they without their IT representation, resources and professionals?
\item
  Can they without their supply chain representation, resources and
  professionals?
\end{itemize}

I think not.

Why then should organizations and their executives think differently
with regard to HR?

If HR is required to justify their existence or inclusion into the C
suite, then something much more serious is wrong and People/HR Analytics
isn't going to fix that. HR not being represented there would be as much
an executive credibility problem as an HR credibility problem.

\textbf{We define our successes or failures the wrong way}

There seem to be at least a couple of widely held misconceptions out
there about how to define success or failure of HR analytics efforts
under the first paradigm (add-on and its assumed need for additional
resources):

\begin{itemize}
\item
  We must find something significant or earth shattering in our efforts
  that will benefit the organization for HR Analytics to have value. And
  we must keep on doing that to justify our existence.
\item
  We must see the various levels of analytics- descriptive, diagnostic,
  predictive and prescriptive as a `maturity' model and strive to be at
  the prescriptive level for HR Analytics to have any value to the
  organization.
\item
  HR Analytics must result in our presence in the c-suite, or must help
  the organization strategically or it doesn't have any value.
\end{itemize}

Its hard to know where to begin with correcting the above
misconceptions.

\begin{itemize}
\item
  At the heart of all statistics and research is determining
  \textbf{whether there are patterns in the data or not}, and if so what
  they are and how strong are they. The reality is that in our data-
  there may not be any significant patterns. But even the absence of
  patterns in the data is still a valuable finding. You now know
  something you did not know before. The value is in the inquiry and the
  research, not on whether there was something earthshaking or not, or
  an answer was found in an organizationally desired direction. When we
  make HR Analytics a `thing' that has to justify its existence through
  earth shattering findings- we have the wrong criteria for success and
  we doom HR Analytics longevity.
\item
  Seeing the level of analytics as a maturity model is erroneous, and
  betrays an accurate understanding of statistics and machine learning.
  There is no such thing as prescriptive being more mature than
  predictive, predictive more mature than diagnostic, diagnostic more
  than descriptive. These levels, at best, represent different levels of
  statistical complexity and skills to understand and use. But each of
  them are specific to the questions asked:

  \begin{itemize}
  \item
    descriptive - what happened
  \item
    diagnostic - why did it happen
  \item
    predictive - what will happen ( assuming a time dimension). ie you
    can predict categories of things by their features.
  \item
    prescriptive - what should happen
  \end{itemize}

  You dont apply a `what should happen' analysis to `what did happen',
  You dont apply a `what will happen' analysis to a `why something
  happened' question. Each type of analysis is suited to answering
  specific types of questions. That is why a `maturity' model
  perspective here is so damaging to evaluating HR Analytics value on
  the basis of it.
\item
  Leveraging HR Analytics to gain entrance to C-suite, or requiring it
  to provide only `strategic' value or it isnt useful- again simply is a
  misundertanding of HR Analytics. \textbf{It exists to serve all HR
  needs.} This includes giving the organization a current
  picture/snapshot of its human resources and trends, operational
  effectiveness and efficiency measurements, AND also to serve future/
  strategic needs None of these are more important than the others. All
  are part of the picture. HR Analytics may be something that assists
  credbility ( because it is data driven and evidence based by it s
  nature). But it should never be held up as a panacea for ongoing
  orgnaization problems that have nothing to do with HR data.
\end{itemize}

\textbf{We start thinking in terms of territoriality}

This happens , in part , again from poor definitions and understanding
of what HR Analytics is.

Sometimes this occurs in turf wars between IT and HR, and this some
times occurs because IT thinks that HR analytics is a data warehousing
business intelligence tools problem and issue.

Sometimes this is combined with a Finance area also not trusting HR to
properly understand the calculation of metrics especially HR ones based
on financial data. So an initiative can end up resulting in little more
than HR dashboards with some self service slice and dice. And it is
presumed by those that created it that HR Analytics has now been
achieved. In reality - a tool has been produced with no guarantee that
it will be used for data-driven decision making. This is kind of like a
-if you build it they will come- mentality. A technology looking for a
use or a customer.

Sometimes in can come in the form of turf wars between decision support
and HR. This often comes when data scientists are seen as a generic
field or job. On social media you often see many people asking the
question `how do I get into the data science field?' as if it were a
separate field independent of a context.

It may be an independent area, study or discipline in academia (and
rightly so). But data science in organizations only has value within a
context of application- whether it be in medical research, industrial
research, organization or operational research to find answers to
critical questions. That `application' requires `domain' knowledge.
Otherwise it is simply a set of tools looking for an application or
customer. It takes some domain knowledge to be in a position to
recognize what the critical questions are to ask, even more domain
knowledge to know what type of data will be either necessary or need to
be available to answer them. And, by the way, if you are proactive as an
HR service, you should often be the ones that generate some of these
critical questions - not waiting for the C suite to ask them.

These turf wars occur when a part is seen as the `whole', or we mistake
having built a tool and conclude we have now achieved analytic success.

The bottom line is that HR Analytics is the `synergy' where HR domain
knowledge comes together with the data tools and the statistical
knowledge necessary of how to turn a business question into acquiring
the needed data and applying appropriate statistical tests and
algorithms. In other words the synergy of IT, Data Science , and HR. In
recognizing that there should be no need for turf wars. The IT tools and
infrastructure, and the data science knowledge in an organization are
useless for HR Analytics without this synergy.

\section{So How Do We Move Towards The Second
Paradigm?}\label{so-how-do-we-move-towards-the-second-paradigm}

I think many things can help here:

\textbf{We need to draw a distinction between the technology and tools
and their complexity and sophistication- and the business questions
being asked, data needed,and statistical analyses and tests required.}

Being data-driven, evidence based, doesnt necessarily require additional
or new resources. Some of the best data science tools are free these
days. In HR, we are awash in HR data that we already have that is
accumulating and never analyzed. \textbf{We aren't making use of what we
already have.}

Want to impress the C-suite? See patterns in your HR data that lead to
cost savings, with what you already have and you will catch their
attention. I have always been a proponent of `skunkworks' (informal and
often behind the scenes) projects in organizations.

If you have been given authority for your role and function within the
organization, its up to you to test out new ideas, and make better use
of what you have. Once you have proven or shown a prototype of something
that works, you can often get further organization support and more
resources for it in a much easier fashion.

The sophistication and complexity of tools isn't and shouldn't be a road
block for HR Analytics.

\textbf{We need to recognize that the roadblocks to HR Analytics are not
predominantly from outside of HR, but from within HR}.

Its quite easy for HR to shift the blame for not being as `data-driven'
and `evidenced-based' as they should/could be:

\begin{itemize}

\item
  `The executive doesnt support this'.
\item
  'We aren't represented in the C-Suite''.
\item
  ``IT and Finance are taking over ownership of the production of all
  metrics, leaving HR out''.
\end{itemize}

Certainly,some of the blame can rest on groups outside of HR. But how
often is that then used as an excuse for `status quo' in HR. None of the
above prevent HR Analytics. `Traditional' HR methodologies and practices
are often the biggest roadblock to `data driven', `evidence based' HR.
This may be because of a number of factors:

\begin{itemize}
\item
  the historical context from which those practices took root in and
  evolved from
\item
  the degree to which HR has become very comfortable over the decades
  with being reactive in the provision of their function- waiting for
  something to happen then have mechanisms policies and procedures in
  place to respond;
\item
  the way in which HR has been taught in educational institutions and
  professional bodies over the decades.
\end{itemize}

\textbf{We need to recognize the need as a profession to maintain
ownership of HR Analytics definitions, taxonomies, concepts with the
help of colleges and universities and require research and evidenced
based backing in these where applicable.}

My sense over 4 decades in HR is that too much of the above and the
resultant HR practices have been driven by vendors selling approaches
and solutions , and not enough by rigorous evidence-based research in
conjunction with universities and colleges.

There should be far more interaction between academia and HR
professional bodies to support the above ownership and drive it forward.
How often are presentations at annual HR professional conferences
dominated by vendors and consultants selling a product vis-a-vis actual
research or examples of new organizational application of HR Analytics
in real life circumstances?

\textbf{We need to recognize that HR Automation and HR Analytics are not
the same thing.}

These are different and both are important. HR Analytics may be assisted
by HR Automation and often this can be very helpful. But the two can
exist relatively independently of each other.

You can have all the automation of HR information in the world and
automation of HR processes and still not have HR analytics (data driven
and evidence based HR management and decision making).

And you can have HR Analytics with little more that a spreadsheet with
the required HR data and statistical software and the knowledge of how
to transform a question into the required analysis to answer it. Most
often we have the data. The tools are free. Whats missing is HR having
an informational/data mindset the statistical knowledge of how to do it.

\textbf{We need to recognize that any long term success and
sustainability for HR Analytics wont come from an `add-on' .}

HR Analytics -to be data driven and evidence based with longevity
requires HR to re-educate and retool itself skills wise - top to bottom.
Without that, it will die quickly because as an add-on \textbf{the
traditional HR functions, practices and methodologies will continue as
they always in their same direction}- no ownership of for `data driven
and evidence based' because thats the responsibility of the `add-on'
resources/team. Every area of HR, every HR function has to take
ownership and responsibility for their practices and methodologies to be
data -driven and evidence-based and obtaining the skills necessary to do
that.

\textbf{We need to recognize that simply collecting and analyzing
`employee' or `applicant' data is insufficient as a scope for HR
Analytics.}

These hardly begin to describe the totality of the landscape of
analytics as applied to HR. Data-driven evidence-based HR Management and
decision making concerns itself with all HR data:

\begin{itemize}
\item
  HRIS data of all kinds, employee data ,position data, job description
  data, applicant data, payroll data, time and labor data
\item
  HR operational data -data related to measuring the activity and
  quality of your HR services
\item
  HR methodologies/practices data- data related to the
  efficiency/effectiveness of how you do what you do -applying AI and
  machine learning directly in these wherever possible.
\end{itemize}
  These go far beyond just analysis of employee data. All of these
  contribute to HR management and decision making and therefore must be
  part of the scope.


\section{What Changes Are Required By The HR
Profession?}\label{what-changes-are-required-by-the-hr-profession}

If you you agree that your choice of paradigm is important, and that the
first paradigm is insufficient or at best just a starting point then
much will probably be required in terms of changes to the HR profession.

\textbf{`Data Driven' and `Evidence Based' HR management and decision
making requires HR to fundamental rethink of how we do the business of
HR.}

This means that all HR practices and methodologies need to be reviewed
in the light of what information they generate, where it stored
(assuming that it is even stored), how those methodologies and practices
can be made to be more data driven. This needs to be done at an
organizational level, and at an academic/teaching level within HR
curriculums at college and universities.

As mentioned previously, some traditional HR functions, methodologies
and practices have been more prone to be somewhat `data-driven' even
traditionally than others. For example the following:

\begin{itemize}
\item
  Collective Bargaining and the costing out of agreements
\item
  salary administration and surveys
\item
  human resource forecasting.
\end{itemize}

But wide swaths of HR methodologies and practices have been either
minimally or `non' data driven for decades. The following come to mind:

\begin{itemize}
\item
  \textbf{Job Analysis and Job Descriptions.} These have been way too
  manual, subjective and non-automated nor data-driven for far too long.
  And this is because the HR profession is locked into traditional
  thinking and traditional methodologies. The entirety of job analysis
  and descriptions can be totally be made to be much more data driven
  and evidence-based by changing our methodologies. And these
  alternative methodologies need to be taught in university and college
  curriculums and be subsequently adopted by HR professionals at an
  organizaional level.
\item
  \textbf{Job Classification.} Those involved in this area of HR- need
  to open themselves to new ways of thiinking and methods that go beyond
  the traditional job classification and valuing schemes found in most
  organizations. Data-driven here means that we recognize that AI,
  machine learning and statistics have a whole lot to say and contribute
  to this area of HR. At the heart of job classification is
  `classiifcation'. There are whole categories of machine learning
  algorithms dedicated to `classification' that have the potential to
  improve both the speed, accuracy and explainability of these decisions
  in organization. The earliest of these is probably 40 years old, and
  many more have evolved in the last 40 years. Why do we see almost no
  evidence of use of these for job classification purposes in
  organization, and little if any understanding of their applicability?
\item
  \textbf{Evaluating Performance.} Traditional performance appraisals
  and 360 degree assessments are not data-driven performance
  measurement. These tools are a subjective assessment and an attempt to
  objectify these assessments with more than one opinion.

  \begin{itemize}
  \item
    Real performance measurement concerns itself with an organizations
    business processes and the improvement of these to achieve better
    products and services (recognizing that we all work within these
    business processes). W. Edwards Deming recognized this decades ago,
    and warned organizations of its dangers and weaknesses in favor of
    continuous process improvement.
  \item
    Why does HR still have an addiction to these? Performance is almost
    always a team effort within existing business processes. When these
    are improved and efficiency gains are made they should be rewarded
    on a team basis. The one by one individual performance appraisals
    approach should often be used only in performance improvement plans
    where often beahvioural issues incompatible with the business
    processes might be what needs to be addressed.
  \end{itemize}
\end{itemize}

\textbf{It also means seeing HR as business within the business- seeing
HR as a provider of services with internal customers that use those
services and with business processes that provide those services.}

A big part of the role of HR Analytics is to measure our HR Services,
business processes and their quality. Reducing the time and cost of
these processes by increasing quality and efficiency is always
consistent and compatible with organizational outcomes. Reducing costs
and improving quality always affects the bottom line. Measuring HR
business activity and business processes are fundamentally data-driven
and evidence based. HR needs to rethink their often current role of
reactive administration and move to proactive continuous improvement.

How many CHROs have systems in place to log every service request that
comes in to HR services and then measures the time taken between receipt
and actually starting it and then the actual time to complete it? These
are the absolute basic minimum data gathering and measurement
requirements to be data driven and evidence based in our HR operations.

Without these, are we as HR really `managing' and providing `leadership'
in HR, or are we `administering, caretaking and reacting' only in HR?

\section{Conclusion}\label{conclusion}

My purpose in this blog article is to have HR as a profession and its
professional bodies consider very carefully their paradigm for People/HR
analytics. Your choice may very well artificially limit its success it
isn't defined or operationalized properly right from the very beginning.
AI, statistics, and machine learning hold so much promise to bring HR
into levels of relevancy and organizational breakthoughs not seen
historically in the profession. But these will occur only to the extent
that HR fundamentally rethinks its role, operations , methodologies and
practices. Traditional HR -business as usual - will not cut it nor bring
these promises or potential to fruition.


\begin{verbatim}
About Lyndon Sundmark, MBA

Lyndon is a retired HR Professional with over 40 years experience
of applying a 'data-driven', 'evidence based' mindset to HR practices
in organizations in a variety of roles and industries.
\end{verbatim}

\end{document}
