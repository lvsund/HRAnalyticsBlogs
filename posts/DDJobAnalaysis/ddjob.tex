\documentclass[12pt,letterpaper]{article}
\usepackage[T1]{fontenc}
\usepackage{graphicx}
\usepackage{parskip}
\usepackage{hyphenat}
\usepackage{url}
%\usepackage{titling}
\title{Data-Driven' Job Analysis And Job Descriptions}
\author{Lyndon Sundmark}
\begin{document}
\maketitle
\section{Data - Driven Job Analysis And Job Descriptions}

\emph{Job Analysis and Job Descriptions in their current form, for the
most part, are \textbf{NOT} data-driven.}

\subsection{Introduction}\label{introduction}

If you have read any of my previous blog articles- but particularly the
last one- you will get the clear impression that I don't see HR
Analytics as an `add-on' to HR. I see it fundamentally as a means to
change the way `we do' HR.

If you accept HR Analytics as `data-driven HR management and decision
making', then it's all about making `traditional' HR Management and
Decision Making- `data-driven'. It means `data-driven' is infused into
\textbf{all} of HR in some fashion and at some level.

This includes Job Analysis and its end results as well.

In this article I therefore wanted to focus on Job Analysis and Job
Descriptions concerning `data-driven'.

The reason for this is because Job Analysis impacts the many other HR
practices that are dependent on it. Being data-driven here enables many
`other' HR practices to start becoming more data-driven.
These HR practices include:

\begin{itemize}
\item
  salary administration
\item
  classification
\item
  recruitment /talent acquisition
\item
  labor relations/ collective bargaining
\item
  training
\item
  career planning
\item
  human resource planning
\item
  occupational health and safety
\item
  and others\ldots{}
\end{itemize}

You can probably regard this article again as an `opinion' piece. Its
intent is to get you to `think differently' about job analysis and job
descriptions.


This blog article will:

\begin{enumerate}


\item
  Revisit what `data-driven' requires and what this might mean when it
  is applied to this area of HR.
\item
  Take a brief look at what makes up much of traditional job analysis
  and its methodologies, and job descriptions content over the last 50
  or more years and how those approaches reflect the larger management
  systems that HR finds itself in.
\item
  Evaluate the degree to which `traditional' job analysis methodologies
  and job descriptions lend themselves to the requirements of
  `data-driven'.
\item
  Explore in practical terms why we should care about `data-driven' in
  job analysis
\item
  Explore an alternative way of thinking and methodology for job
  analysis and descriptions that is `data-driven'.
\item
  Show what new possibilities `data-driven' Job Analysis and
  Descriptions give us in terms of truly transforming the work of HR in
  the organization.
\end{enumerate}
\subsection{Revisiting What ``Data-Driven''
Means}

``Data - Driven'' is at the heart of what HR Analytics is all about -
`Data - driven' HR management and decision making . We are interested in
\textbf{`evidence-based'} management and decision-making regarding
people.

This means at least three things:

\begin{itemize}
\item
  committing to making more HR decisions based on objective information
  and more rigorous analysis.
\item
  organizing ourselves around making `that' objective information more
  visible and accessible in our HR practices
\item
  keeping that information in a form that readily lends itself to
  automation and analysis. This requires the information to:

  \begin{itemize}
  \item
    exist
  \item
    be standardized
  \item
    be structured
  \item
    be quantifiable or categorizable
  \item
    be stored in a database
  \item
    allow for the summary, comparison, and contrast - readily and quickly
    across the entire set of data.
  \item
    allow for appropriate statistical analyses and reporting to answer
    questions posed of that data.
  \end{itemize}
\end{itemize}

\subsection{Revisiting Traditional Job Analysis And Job Description
Approaches}

To answer this question, it is always helpful to look at a variety of
definitions from a variety of different sources, to get a sense of the
commonalities that emerge.


A quick and cursory Google search turned up many definitions.

%\href{https://www.google.com/search?q=job+analysis+definition&rlz=1C1CHBF_enCA922CA922&oq=job+analysis+&aqs=chrome.0.69i59l2j69i57j0i67i395l5.7991j1j15&sourceid=chrome&ie=UTF-8}{Job Analysis Definition Google Search}

One of them that came up in my Google search was from Wikipedia:

\textbf{Wikipedia}

\url{https://en.wikipedia.org/wiki/Job_analysis}

Job analysis (also known as work analysis is a
family of procedures to identify the content of a job in terms of
activities involved and attributes or job requirements needed to perform
the activities.)

Job analysis provides information of organizations which helps to
determine which employees are best fit for specific jobs. Through job
analysis, the analyst needs to understand what the important tasks of
the job are, how they are carried out, and the necessary human qualities
needed to complete the job successfully.

The process of job analysis involves the analyst describing the duties
of the incumbent, then the nature and conditions of work, and finally
some basic qualifications. After this, the job analyst has completed a
form called a job psychograph, which displays the mental requirements of
the job.The measure of a sound job analysis is a valid task
list. This list contains the functional or duty areas of a position, the
related tasks, and the basic training recommendations. Subject matter
experts (incumbents) and supervisors for the position being analyzed
need to validate this final list in order to validate the job
analysis.

I wont profile all the google hits above in this blog article, but if
you look at many of them they are identical or very similar- and have
many commonalities between them.

What are some of these commonalities?

They:

\begin{itemize}
\item
  study the \textbf{job} to identify its content which can include:

  \begin{itemize}

  \item
    activities
  \item
    responsibilities
  \item
    qualifications
  \item
    working conditions
  \item
    tasks
  \item
    duties
  \item
    knowledge
  \item
    skills
  \item
    abilities
  \item
    importance of tasks
  \item
    human qualities
  \item
    how often tasks are performed
  \end{itemize}
\item
  have multiple purposes in mind for the usage of the gathered
  information.
\item
  have a job description often as the most immediate end result.
  Typically this \textbf{written narrative} describes tasks, duties and
  responsibilities of a position. This relationship between job analysis
  and job description is described as well in Wikipedia:
\end{itemize}

\begin{itemize}
\item
Wikipedia definition of a job description 

\item "According to Torrington, a job description is usually developed by conducting a job analysis, which includes examining the tasks and sequences of tasks
    necessary to perform the job. The analysis considers the areas of
    Knowledge "Knowledge",
    "Skills", and 
    abilities"
    needed to perform the job. Job analysis generally involves the following
    steps: collecting and recording job information; checking the
    job information for accuracy; writing job descriptions based 
    on the information; using the information to determine what skills,
    abilities, and knowledge are required to perform the job; updating the information from time to time."

\item Historically-once produced- these job descriptions would be stored 
in filing cabinets for retrieval when needed.
    In the 1980s and later these would be 'Word' documents stored on a
    network file server somewhere.
    Some organizations took it a step further and put them in HTML format
    on web server.
    Regardless though, these job descriptions are still 'narrative documents'
    in whatever technology form they're in.
\end{itemize}

The above methods and formats have been in the literature, textbooks and
professional HR education for 40-50 years with very little change. Two
key questions at this point might be:

\begin{itemize}
\item
  why is this the past and current paradigm for job analysis?
\item
  Does this paradigm lend itself to being data-driven?
\end{itemize}

We will explore the first of these questions in this section immediately
below and the second in the next.

\textbf{Why is this the past and current paradigm for job analysis?}

To answer this, we have to step back a bit and realize that the
\textbf{job} and/or the \textbf{position} is and has almost always been
the focal starting point of all \textbf{job analysis}.

From above:

\emph{``\textbf{Job} analysis (also known as work analysis{[}1{]}) is a
family of procedures to identify the content of a \textbf{job} in terms
of activities involved and attributes or job requirements needed to
perform the activities.}

\textbf{Job} analysis provides information of organizations which helps
to determine which employees are best fit for specific \textbf{jobs}.
Through \textbf{job} analysis, the analyst needs to understand what the
important tasks of the job are, how they are carried out, and the
necessary human qualities needed to complete the \textbf{job}
successfully.''

Notice that the focus is the \textbf{JOB .}

In this paradigm, if you wanted to understand the organization and
document its work, you study and document the jobs - one by one.
The assumption is that the 'work of the entire organization is really
seen as the aggregate of all of these together. 

You start at theindividual level and you `aggregate' to get the 
entire picture of an organization's work. This is a `proxy' for 
understanding the business that the organization is in. The assumption here is aggregating from individual jobs and positions will synthesize
into an overall `coherent' picture of the work of the organization.

\underline{This `job by job' analysis paradigm} is not independent of the other
organization management paradigms they are within. To map out and to
know when you have the work of the entire organization covered, you
traditionally have depended on organization charts. When you have
covered every position/job in the organization chart you supposedly have
the work of the organization documented.

That got me doing a little detective work on how organization charts
originated and why they are so pervasive. This would be a whole other
topic in itself which I won't go into detail here. But that minor
detective work led me to the following book:

%\href{https://www.amazon.ca/Leaders-Handbook-Making-Things-Getting/dp/0070580286/ref=sr_1_1?dchild=1&keywords=peter+scholtes&qid=1612292165&sr=8-1}{The
%Leaders Handbook -Peter Scholtes}

In that book, Peter indicates that organization charts came from
`train-wreck' charts (an apt metaphor? ;) tongue in cheek). The intent
was to prevent train wrecks that occurred in the 1800s from happening in
the future. These `train-wreck' charts among other things illustrated:

\begin{itemize}
\item
  chains of commands
\item
  functional divisions
\item
  clear descriptions of responsibility
\end{itemize}

In his book, Peter goes through a chronological analysis of how the
above eventually resulted in Management By Objectives (MBO) as a system
of management.

Out of those larger management systems such as MBO would have likely
come the HR methodologies for job analysis. 
\underline{i.e.clear descriptions  of responsibility} sounds an awful
lot like job descriptions to me.




The point I am raising is that our HR practices didn't emerge in a
vacuum. They were responses intended to be consistent with the larger
management systems they found themselves within.

These larger management systems aren't `without' impact or consequences.

In MBO, the consequence is that if you want to `understand' a business,
you look at an organization chart. You look at the chain of commands,
functional divisions, and descriptions of responsibility. For what it's
worth, personally I never found that to be true. Looking at the above
helped me understand how the business was organized or structured - but
that isn't the same as understanding the business.

In HR, the impact was `job descriptions' and a methodology for job
analysis that was `one by one' and with the assumption that aggregating
from the `bottom up', you would then have a picture of the work of the
organization and the business it was in. The `one by one' ultimately
weaves itself into traditional performance appraisal systems. We
evaluate the performance of employees `one by one'. MBO requires it, and
traditional job analysis and descriptions support it -`one by one'.

The `bottom up', `one by one' paradigm rarely leads to a coherent
picture of the work of the organization and the business it is in- more
often than not it is an incoherent, fractured picture.

Let me illustrate why.


\textbf{Picture Puzzle Analogy}

Most of you are familiar with picture puzzles (a \textbf{picture} that
has been carved up into several hundred pieces). You probably grew up
with them as kids. The reason why these puzzles work is that they start
with a `whole picture' and carve it up into pieces. When you put those
pieces together they from the whole original picture.

Now imagine instead that you have half a dozen people independently who
are attempting to create puzzle pieces from scratch. Even if they know
that the end result is `a picture' (hopefully a coherent business
purpose), their differences (as human beings) will result in a variety
of puzzle pieces which are unlikely to fit together well or at all.

When you start with the pieces, rather than the entire picture, those
pieces will find it almost impossible to form an entirely coherent
picture. There is very little commonality between those pieces.

I would suggest that traditional job analysis methods are a lot like
that. You may have all the pieces, but you have very little way of
linking them together- finding pieces that fit and join together. We are
so focused on the \textbf{`job'} as the entity under study that we cant
see the larger picture under this paradigm. (It's also noteworthy in the
above wikipedia definition it said `also known as work analysis'. I
think we have forgotten that.)

\textbf{We need a paradigm with job analysis methodologies and end
results (job descriptions) that start with the overall picture and
deliver puzzle pieces that stand on their own - AND still fit back
together as a unified whole.}

Being `data-driven' in our job analysis methodologies and job
description results is a potential enabler in this direction.



\subsection{3.Evaluating The Degree To Which Traditional Job Analysis
and Descriptions Meet The Criteria Of Data-Driven}

Let's look at the characteristics of `data-driven' and evaluate the
degree to which traditional job analyses and descriptions meet these
requirements.

\subsubsection{\textbf{3.1 Information Exists (Requires
Data orInformation)}}

Loosely speaking- even traditional job analyses and descriptions meet
this criterion. Traditional job analyses are a manual process for
`generating' \textbf{information} about the contents of a job. And as
the methods mentioned above indicate- there are quite a few ways of
generating that information. \textbf{PASS}
\subsubsection{\texorpdfstring{\textbf{3.2 The Data or Information Needs
To BeStandardized}}

Loosely speaking we could say that this requirement is also met (at
least in part). Our study of jobs through job analysis focuses on
\textbf{standardizing the content}. This is reflected above in what the
typical content areas are:

\begin{itemize}

\item
  activities
\item
  responsibilities
\item
  qualifications
\item
  working conditions
\item
  tasks
\item
  duties
\item
  knowledge
\item
  skills
\item
  abilities
\item
  importance of tasks
\item
  human qualities
\item
  how often tasks are performed
\end{itemize}

\textbf{If} you cover some or all of these `consistently' in every job
analysis and resulting description you are `standardizing' your
`approach' to the \textbf{content}. (but maybe not the expression of
that content)

At least at a `narrative' level then most organizations achieve
standardization through guidelines and procedures. Undoubtedly though,
some variation will occur in the standardization due to different people
with different writing styles. \textbf{PASS}

\subsubsection{\texorpdfstring{\textbf{3.3 The Data or Information Needs
To BeStructured}}
This means:

\begin{itemize}
\item
  not only are the contents standardized but they also
\item
  are expressed in a way that we can easily see and find the equivalent
  information from job to job.
\end{itemize}

This will be a \textbf{PASS} most of the time if the narrative
descriptions are standardized and well written. \textbf{But in my
opinion -a bare pass} because there can be a lot of subjectivity and
judgment around the equivalence around the job content and
characteristics.

Those of you who have done job classification by comparing job
descriptions to `narrative' job class specifications could probably
attest to the fact that even when policies, procedures, and guidelines
are in place for writing job descriptions- the ease of slotting it into
the most appropriate job class is not easy. It is definitely partly a
function of how well the job description was written in terms of
structure and standardization and the job class specifications as well.

In other words, even with the best of intentions and guidelines for
writing, the use of traditional narrative job descriptions can be very
difficult.

\subsubsection{\texorpdfstring{\textbf{3.4 The Data or Information Needs
To Be
Quantified}}{3.4 The Data or Information Needs To Be Quantified}}\label{the-data-or-information-needs-to-be-quantified}

Because most traditional job analyses and descriptions end up being
narrative documents derived from narrative information gathering,
\textbf{this requirement is typically not met}. At best there may be
some quantification of time spent (i.e.~proportion of time spent) on
major responsibilities. But for such things as skills, knowledge, and
abilities required, these may be vague statements of whether there is
more of something or less of something required, and these are usually
`narrative' -\textbf{NOT quantified}.

Even if minimum information is quantified, if the structuring of job
information or the standardization of it is weak, quantification in
these circumstances doesn't help us much. We really don't know if X\%
time spent on Y responsibility is really comparable to Z\% time on the
same responsibility in another job.

Without strict attention to standardization- we may or may not be
comparing the same responsibility. \textbf{True standardization means
that we capture the same types of common data on EVERY JOB.} Narrative
job descriptions (unless heavily standardized, structured, and written
with an eye to eventual quantification of that information) rarely meet
this criteria. \textbf{FAIL}

\subsubsection{\textbf{3.5 The Data or Information Needs
To Be Stored In A
Database}

\textbf{Right from the `get-go' this requirement in traditional job
analysis and descriptions is rarely, if ever, met.} Combing and scouring
narrative sources of written information, verbal content from
interviewing Subject Matter Experts (SMEs), etc. is not in a format
needed for storage in a database. Automating job descriptions to the
point of being either Word documents on a file server or HTML documents
on a web server for easy access \textbf{isn't `being stored in a
database'.}

For job analysis information and resulting descriptions to be stored in
databases:

\begin{itemize}
\tightlist
\item
  Each job must be stored as a single entity (often a record) in the
  database. (often somewhat comparable to thinking of a row in Excel)
\item
  All the information used to describe that job and its characteristics
  must be represented by \underline{data elements or fields} for that record or
  sub-records. (often comparable to thinking columns in Excel for a
  single row)
\item
  The information must be structured and standardized in the database.
  (In Excel, all columns appear in every row)
\item
  If information is to be used to compare one job to another, all
  information to be used for comparison must be categorized into
  standardized qualitative information if textual (ie non-numeric) or
  quantified into measures if numeric. The reason for this is to have
  common denominators on which to compare and contrast different jobs.
\item
  The database must hold the information on all documented jobs.
\end{itemize}

The verdict on `traditional' job analysis and descriptions:
\textbf{FAIL}

\textbf{Nothing is preventing an organization from thinking
non-traditionally here}. The idea of storing job information in a
structured, quantifiable, measurable way is \textbf{NOT NEW}. I am not
the originator of this idea. Early attempts to use the technology to do
this are at least 30 years old if not older. Even 30 years ago, it gave
me some sense of the future possibilities and how things could change if
we wanted them to.

So why no huge uptake?

A lot of the answer to that goes back to the difference that HR
Analytics requires. `Data-driven' HR Management and Decision Making
absolutely requires structured and standardized information and
measurement. Traditional HR doesn't. It's why the literature and methods
around this part of HR haven't changed much for 50 years. I'm not sure
that's a good thing.

\subsubsection{3.6 The Data or Information Needs
To Readily Allow for Quick Contrast, Comparison, and Summary Across The
Entire Dataset (i.e.~ ALLJOBS)}


This too is an immediate \textbf{FAIL} on the criteria of `data-driven'.

Word documents on a file server or HTML documents on a web server do not
allow for any quick comparison or contrast across all jobs. At best Word
documents would be a manual effort of perusal, and HTML documents would
offer keyword search capabilities. But there is no ability to
quantitatively compare information across all jobs.

\textbf{So, on balance, how do traditional job analysis and job
descriptions stack up against `Data-Driven' HR?}

\textbf{Poorly.}

\subsection{4. Why Should We Be Concerned With `Data-Driven' In Job
Analysis and Job Descriptions? Why Is This
Important?}

I think the answer to this comes down to a few major reasons.

\begin{itemize}
\item
  Traditional job analysis and descriptions and their methodologies make
  the process of doing this far more manual than it needs to be and far
  longer than it needs to be.
\item
  Some things in HR are way more difficult to do or do efficiently and
  well, or do at all -with traditional job analysis and descriptions.
\item
  As mentioned earlier there are many uses made of job information in
  HR, so by making improvements here, we improve many areas of HR.
\end{itemize}

Let's look at each of these reasons in turn.

\subsubsection{4.1 Traditional Job Analysis and Descriptions-Too Time
Consuming}

Those of you whose job is job analysis can probably attest to this.
Traditional Job Analysis and Job Descriptions are a laborious, manual,
and slow processes- whether its in the creation of these or usage of
them when job descriptions are done.

\textbf{For creation:}

\begin{itemize}
\item
  a request for a new job is created.
\item
  you will likely be required to interview the person to who this job
  will report to.
\item
  you will likely need to interview SMEs if there are any
\item
  if it's an existing job that is changing- you will likely have to
  interview an incumbent
\item
  you will likely have to review your organizational standards for job
  analyses and descriptions
\item
  you will likely have to review other jobs to understand how this is
  similar or different from them
\item
  you will prepare new job description from all your analysis
\end{itemize}

Best guess-a few days to organize all of this into an end result.

\textbf{For HR usage:}

Let's use job classification (using narrative job class specs) as an
example:

\begin{itemize}
\item
  you will need to read the job description in question- getting some
  sense of its content.
\item
  if you are a season classification officer, you will form an initial
  impression of the job family it falls into.
\item
  you will then need to review at least a few job classification specs,
  and perhaps many other comparable positions/jobs to ballpark your
  decision on classification.
\item
  you meet with the requester and present findings and hopefully justify
  the decision made to their satisfaction.
\end{itemize}

Best guess-a few days to organize all of this into a completed decision.

\ul{What if being `data-driven' in our approach resulted in job analysis
and descriptions taking no more than a couple of hours? What if upon
obtaining the information- classification took only minutes?}

\emph{Do we in HR actually care about increasing the efficiency of our
methods, procedures, and techniques and resulting customer response
time?}

\subsubsection{4.2 Some Things We Do In HR Much More Difficult If Not
Impossible Using Traditional Job Analysis and Description
Approaches}

There are at least a few examples here that come to mind.

Consider the following:

\subsubsection{An example}\label{an-example}

\begin{itemize}
\item
  Strategic planning, career planning, and succession planning are all
  dependent on good job information and to be able to use that
  information in a specific way. The common denominator in all these
  planning efforts is the the successful `defining and bridging of
  gaps'.

  \begin{itemize}
  \item
    For strategic planning it's defining the gap between who, where, and
    what you are now as a business and the desired future for your
    business- desired by you as an organization and by your customers -
    and bridging that gap with plans and activities. ( strategic
    planning is really career planning at a `business entity' level).
  \item
    For career planning, it's defining the gap between who you are and
    where you are right now and what you desire to be in the future and
    then bridging that gap with plans and activities. (career planning
    is really strategic planning at an `individual' level)
  \item
    For succession planning, it's defining the gap between job
    requirements needed for the future, and succession candidates
    current capabilities and coming up with developmental plans and
    activities to reduce or bridge that gap.
  \end{itemize}
\end{itemize}

Doing any/all of these, at some point, requires us to bring this down to
a job content, job analysis, job description level. To truly get down to
the nuts and bolts of defining gaps- you have to define differences. To
define differences, you must be expressing things in a standardized way
that allows you to say for instance:

\begin{itemize}
\item
  these items belong to job 1 and only job 1
\item
  these items belong to job 2 and only job 2
\item
  these items belong to both job 1 and job 2
\item
  to get from job 1 to job 2, the person must keep all things common to
  both and learn items in job 2
\end{itemize}

The same goes for strategic planning:

\begin{itemize}
\item
  These activities/products/services are current and only current
\item
  These activities/products/services are what we desire for future and
  are only future
\item
  Some activities/products/services will be both current and continue
  into future
\item
  To get from current to future successfully we will need to drop some
  activities /products /services, understand that some will continue to
  need to be maintained, and some new initiatives will need to be
  introduced and plan for all of these.
\end{itemize}

Even strategic planning eventually has to be expressed in terms of how
it affects jobs, their contents, and differences.

If your job analysis methodologies and job descriptions don't quantify,
standardize, structure job information and stored in a database -
allowing for the summary, comparison, and contrast readily and quickly
across the entire set of data- the quality of the above planning
processes and their implementation may be limited.

\textbf{Are your existing traditional job analysis methods and
traditional job description format and contents capable of this and up
to the challenge?} `Data-driven' in this area of HR is likely to be an
enabler for this.

\subsubsection{Another Example}\label{another-example}

If you wanted to extend the strategic planning example:

At some point in strategic planning, as per the above, you should;

\begin{itemize}
\item
  identify the products and services you are no longer engaging in
\item
  identify the products and services that you are going to continue to
  engage in
\item
  identify the new initiatives, products, and services that will come on
  stream
\end{itemize}

Each of these steps requires understanding how this will affect your
current jobs and the employees that are in them.

\textbf{Do we know how many FTEs will be saved by those products and
services no longer engaged in and exactly where?}

\textbf{Do we know how many FTEs will need to continue to support
continued products and services and exactly where?}

\textbf{Do we know how many FTEs will be needed to properly support new
initiatives and exactly where?}

Without good, robust, and data-driven job analysis methodologies and job
description content and storage it's very difficult for HR to answer
these questions and to do it promptly.

\subsubsection{4.3 Data-Driven in our Job Analysis Methods / Job
Description Content Will Improve Many Areas of
HR}

The primary goal of analytics, and for our purposes HR analytics- is to
improve performance in the organization whether it be through

\begin{itemize}
\item
  HR metrics- so that the organization knows what is happening to its
  human resources over time and can be more proactive
\item
  HR Services- being much more aware of the level and quality of
  services we provide to our organizational customers and making
  improvements as a result of that
\item
  HR methodologies- efficiency and effectiveness- to do what we do
  better and quicker (without sacrificing better)
\end{itemize}


Job content is interwoven and related to all of these.

In short- THIS IS WHY WE SHOULD CARE ABOUT DATA-DRIVEN JOB ANALYSIS AND
DESCRIPTIONS.

\subsection{5. What Is An Alternative Way Of Thinking About Job Analysis
and Job
Descriptions?}
\underline{The short answer is to change our paradigm} and its associated
methods, increase our use of technology, and quantify our job
information.

Here are some practical ways to change our `50-year-old' thinking in
this area and move on from historical and current approaches.
\begin{itemize}
\item
  Our job analysis methods need to start with the entire picture and
  work back from there.

  \begin{itemize}
  \item
    The concentration/focus is not on `job' analysis but
    \underline{\textbf{\emph{`work' analysis.}}}
  \item
    Jobs just happen to be the way we segment out that work into
    understandable and manageable chunks for a position or person.
  \item
    any one job is just a subset or snapshot of the much larger picture
    of the work of the organization.
  \item
    As a subset of a larger whole- differences and similarities are more
    readily discernible
  \end{itemize}
\item
  Understand that starting point of understanding the entire picture of
  the work of the organization \textbf{can't be} the organization chart
  (aka train wreck chart). An organization chart tells you nothing about
  the work of the organization, only how you are structured. The
  starting point is:

  \begin{itemize}
  \item
    Confirm what business the organization is in.
  \item
    Then identify and enumerate all of the products and services the
    organization produces/performs in being in that business
  \item
    Then the identification and enumeration of the business processes
    that produce those products and services
  \item
    Then identification and enumeration of the tasks, knowledge, skills,
    qualifications necessary to perform those business processes.
  \item
    Finally determine how to apportion those tasks, knowledge, skills,
    and qualifications into individual jobs and positions. A `job'
    description then is the documentation of assigning tasks and
    knowledges within the business processes that lead to the provision
    of products and services to those individual jobs and positions.
    That way you can go both directions (up and down) in job
    analysis/work analysis for the organization. You `could' represent
    these via an organization chart- but you would probably find it
    equally and possibly more useful to represent your organization via
    business process charts and identify (in some manner) how the
    positions are integrated into them. Each job is a `subset' of those
    tasks and knowledges.
  \end{itemize}
\item
  Understand what measurement means in relation to job information and
  job analyses and realize much is measurable and quantifying that job
  information. This includes:

  \begin{itemize}
  \item
    understanding the presence or absence of tasks, knowledge, and
    skills, qualifications in any `one' job out of the total picture the
    tasks knowledge, skills and qualifications needed is a form of
    measurement
  \item
    understanding that capturing of `levels' of knowledge, skills, and
    qualifications on some sort of scale is in fact -measurement.
  \item
    understanding that time spent on various tasks is measurement
  \end{itemize}
\item
  Understand that when you standardize the job information and quantify
  it, you are enabling that information to be more readily stored in a
  database.
\item
  Understand that when job information is structured and standardized in
  a database, this is what enables much more extensive and robust use of
  that information.
\item
  \textbf{Understand that when job information is structured,
  standardized, and quantified in a database that web applications can
  be developed to serve out much of the functionality expected from the
  traditional approaches but also `much more'.} It's actually a two way
  street here- if we change to a new paradigm for job analysis- the
  tools supporting it must change. Traditional job analysis
  fundamentally requires job descriptions, paper documents, word docs on
  a file server, or HTML docs on a web server. Data-driven requires
  databases and computer applications to manage, house, process an
  analyze the job information.
\end{itemize}

\subsection{What Are New Possibilities When We Become More
Data-Driven in Job Analysis and Job
Descriptions?}

\subsubsection{Administrative Related}

\begin{itemize}
\item
  Job Descriptions that are dynamically kept up to date. Storage in a
  database means they are as current as the last time they were update
  and from a central source
\item
  Job Descriptions that are often available in a variety of formats-
  HTML, Word, PDF etc
\item
  Job Descriptions that are keyword searchable
\item
  Job Descriptions that can be created from online wizards that guide
  the process or printed questionnaires to select job content from.
\item
  Security separation from those who are consumers of job information
  from those responsible for creating the job content in web
  applications.
\end{itemize}

\subsubsection{6.2 More Organizational Impact
Related}

\begin{itemize}
\item
  Better organizational HR planning processes - through detailed
  defining gaps/differences/ similarities between positions for planning
  purposes including career planning and succession planning
\item
  Better organizational strategic planning processes - by helping the
  organization understand the effect on existing FTEs of dropping,
  maintaining, and adding new product and service expectations will have
  on the organization.
\item
  Much more robust HR Management and Decision-Making processes - by
  putting job information more in the form necessary to be able to be
  used in machine learning and artificial intelligence to complement the
  current strengths of this in the organization.
\item
  Better HR and organization performance through recognizing that
  organizing job information around products/services and their
  corresponding business process provides a paradigm for naturally
  gravitating the organization in the direction of ongoing continuous
  improvement.
\end{itemize}

\subsection{Conclusion}

As mentioned earlier, my intent in this blog article is to stimulate
`thinking differently' as to how we understand this area of HR
\textbf{and} give some thought as to whether we should operationalize it
differently in our HR practices. Are we motivated as HR professionals to
do that?

I have often had the thought that many of us might unintentionally be
doing a `Rip Van Winkle' when it comes to some of our HR practices and
methodologies. We could go to sleep for 50 years, wake up, and still do
HR work with our skill sets because so little changes.

The changes that `do occur' are often externally imposed on the
organization and HR profession in the form of meeting changes in
legislative regulatory requirements. It often isn't driven with an eye
of stepping back and asking fundamental questions about our practices
and methodologies on how we could do them better.

\begin{itemize}
\item
  How often do we go to annual HR conferences with an intent of and an
  eye toward truly learning new skills and applying them to change the
  way we conduct the business of HR vis-a-vis seeing the latest vendor
  options that promise to do things for us but require no changes in our
  skill sets?
\item
  How often does HR complain about not being taken seriously or being
  represented at the executive table?
\end{itemize}

As much as I believe that HR representation is essential at the
executive level, it's much more likely that representation will be seen
as more credible when we actually exhibit leadership- through continuous
improvement of our HR practices and methodologies, and enable the whole
organization to perform better as a result. \textbf{Continuous
improvement is intentional and proactive}. Too many traditional HR
practices are reactive, and demonstrate little, if any, innovation.

Technology can support innovation in our HR practices. In fact it's
increasingly fundamental to it. Change in our paradigm will require
changes in methodologies and supporting tools-technological or
otherwise. The biggest obstacle to innovation is often simply being
willing to `think differently'.

`Data-driven', `evidence-based' HR management and decision making (HR
Analytics) helps to mitigate complacency with the status quo. And
changing our paradigm of Job Analysis and Descriptions is also very much
part of that picture.

Food for thought 
\subsection{Quick Addendum- For Anyone
Interested}\

As I mentioned above job analysis and description rely on methodologies
and tools regardless of traditional or new paradigms.

I have been prototyping a `proof-of-concept' web application and
database for data-driven job analysis as a hobby. (Part of my software
development interests and skills). The prototype is being written using
Microsoft's .Net technologies in C\#. If and when I get most of the base
functionality working, I might put it out in the public domain
under the GPL license on Github.com. The intent would be to illustrate
one way of how a new paradigm could actually be operationalized for
use in job analysis and descriptions.

If I succeed in that, the documentation of that `proof of concept' would
be the subject of a possible future blog article.







\end{document}